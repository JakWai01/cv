%%%%%%%%%%%%%%%%%%%%%%%%%%%%%%%%%%%%%%%%%
% Developer CV
% LaTeX Template
% Version 1.0 (28/1/19)
%
% This template originates from:
% http://www.LaTeXTemplates.com
%
% Authors:
% Jan Vorisek (jan@vorisek.me)
% Based on a template by Jan Küster (info@jankuester.com)
% Modified for LaTeX Templates by Vel (vel@LaTeXTemplates.com)
%
% License:
% The MIT License (see included LICENSE file)
%
%%%%%%%%%%%%%%%%%%%%%%%%%%%%%%%%%%%%%%%%%

%----------------------------------------------------------------------------------------
%	PACKAGES AND OTHER DOCUMENT CONFIGURATIONS
%----------------------------------------------------------------------------------------

\documentclass[9pt]{developercv} % Default font size, values from 8-12pt are recommended

%----------------------------------------------------------------------------------------

\begin{document}

%----------------------------------------------------------------------------------------
%	TITLE AND CONTACT INFORMATION
%----------------------------------------------------------------------------------------

\begin{minipage}[t]{0.45\textwidth} % 45% of the page width for name
	\vspace{-\baselineskip} % Required for vertically aligning minipages
	
	% If your name is very short, use just one of the lines below
	% If your name is very long, reduce the font size or make the minipage wider and reduce the others proportionately
	\colorbox{black}{{\HUGE\textcolor{white}{\textbf{\MakeUppercase{Jakob}}}}} % First name
	
	\colorbox{black}{{\HUGE\textcolor{white}{\textbf{\MakeUppercase{Waibel}}}}} % Last name
	
	\vspace{6pt}
	
	{\huge Software Engineer} % Career or current job title
\end{minipage}
\begin{minipage}[t]{0.3\textwidth} % 27.5% of the page width for the first row of icons
	\vspace{-\baselineskip} % Required for vertically aligning minipages
	
	% The first parameter is the FontAwesome icon name, the second is the box size and the third is the text
	% Other icons can be found by referring to fontawesome.pdf (supplied with the template) and using the word after \fa in the command for the icon you want
	\icon{MapMarker}{12}{Stuttgart}\\
	\icon{At}{12}{\href{mailto:salet@posteo.net}{salet@posteo.net}}\\	
\end{minipage}
\begin{minipage}[t]{0.275\textwidth} % 27.5% of the page width for the second row of icons
	\vspace{-\baselineskip} % Required for vertically aligning minipages
	
	% The first parameter is the FontAwesome icon name, the second is the box size and the third is the text
	% Other icons can be found by referring to fontawesome.pdf (supplied with the template) and using the word after \fa in the command for the icon you want
	\icon{Globe}{12}{\href{https://jakobwaibel.com}{jakobwaibel.com}}\\
	\icon{Github}{12}{\href{https://github.com/JakWai01}{github.com/JakWai01}}\\
	%\icon{Twitter}{12}{\href{https://twitter.com/@alyxvance}{@alyxvance}}\\
\end{minipage}

\vspace{0.6cm}

%----------------------------------------------------------------------------------------
%	INTRODUCTION, SKILLS AND TECHNOLOGIES
%----------------------------------------------------------------------------------------

\cvsect{Who Am I?}

\begin{minipage}[t]{0.5\textwidth} % 40% of the page width for the introduction text
	\vspace{-\baselineskip} % Required for vertically aligning minipages
    While having a wide range of interests, one constant interest of mine is computer science. Whether theoretical computer science, algorithms or software engineering, I am always prone to getting nerd-sniped within minutes of hearing to ramblings about a new topic. 
\end{minipage}
\hfill % Whitespace between
\begin{minipage}[t]{0.4\textwidth} % 50% of the page for the skills bar chart
	\vspace{-\baselineskip} % Required for vertically aligning minipages
	\begin{barchart}{5.5}
		\baritem{Rust}{70}
		\baritem{Go}{60}
        \baritem{C++}{40}
		\baritem{Linux}{70}
        \baritem{DevOps}{50}
	\end{barchart}
\end{minipage}

%\begin{center}
	%\bubbles{5/Jet Brains, 4/git, 4/Office, 5/Adobe Suite, 4/Admin}
%\end{center}



%----------------------------------------------------------------------------------------
%	EXPERIENCE
%----------------------------------------------------------------------------------------

\cvsect{Experience}

\begin{entrylist}
	\entry
		{01/23 -- now}
		{Software Engineer at IBM Quantum\\\footnotesize{Working Student}}
		{IBM Germany Research \& Development GmbH}
		{As part of the Qiskit Runtime team, I worked on developing the classical runtime environment of quantum computers.
        \\ \texttt{Go}\slashsep\texttt{Kubernetes}\slashsep\texttt{Containers}\slashsep\texttt{Linux}}
	\entry
		{03/22 -- 12/22}
        {Software Engineer at IBM Data \& AI\\\footnotesize{Intern and Working Student}}
        {IBM Germany Research \& Development GmbH}
		{During my internship, I worked on an open source CLI that allowed deploying our product on kubernetes clusters. After the internship, I maintained clusters and worked on Watson Knowledge Catalog. 
		\\ \texttt{kubernetes}\slashsep\texttt{python}\slashsep\texttt{java}}
	\entry
		{03/21 -- 02/22}
        {Computer Science and Mathematics Tutor}
        {Stuttgart Media University}
		{I tutored students in software development I and applied mathematics.
		\\ \texttt{java}\slashsep\texttt{linux}\slashsep\texttt{probability theory}}
\end{entrylist}

%----------------------------------------------------------------------------------------
%	Projects
%----------------------------------------------------------------------------------------

\cvsect{Projects}

\begin{entrylist}
	\entry
		{2024}
		{Ship Routing}
		{University of Stuttgart}
		{This project required the implementation and optimization of street routing algorithms while applied to ship-routing on the sea. I chose to implement the graph generation using quad-trees and the routing using a combination of contraction hierarchies and hub labels.
		\\ \texttt{Rust}\slashsep\texttt{Algorithms}\slashsep\texttt{Theoretical Computer Science}}	
	
	\entry
        {2022 -- now}
        {Lurk - A (pretty) Simple Alternative to strace}
        {Open Source}
        {The goal of this project was to develop a refined version of strace that allows parsing output via \texttt{jq} while, instead of strace, formatting the output in a human-readable form.
        \\ \texttt{Rust}\slashsep\texttt{SystemsProgramming}\slashsep\texttt{OpenSource}}
        
    \entry
        {2023}
        {Development of Green Guardian}
        {Stuttgart Media University}
        {This project was part of a course about embedded software-development for cloud computing, implemented utilizing AWS services. My responsibilities included designing the system architecture, provisioning infrastructure using Terraform, as well as setting up CI/CD pipelines.
        	\\ \texttt{aws}\slashsep\texttt{terraform}\slashsep\texttt{go}\slashsep\texttt{arduino nano}\slashsep\texttt{mqtt}}
\end{entrylist}

%----------------------------------------------------------------------------------------
%	EDUCATION
%----------------------------------------------------------------------------------------

\cvsect{Education}

\begin{entrylist}
	\entry
        {2023 -- 2025}
        {Master of Science: Computer Science}
        {University of Stuttgart}
        {Algorithms, Distributed Systems, Embedded Systems}
	\entry
        {2020 -- 2023}
        {Bachelor of Science: Computer Science and Media}
        {Stuttgart Media University}
        {Distributed Systems, Data Mining, Software Development, Design Patterns}
	\entry
        {2019-2019}
        {Bachelor of Science: Physics}
        {University of Stuttgart}
        {Linear Algebra, Analysis, Experimental Physics}
\end{entrylist}

%----------------------------------------------------------------------------------------
%	ADDITIONAL INFORMATION
%----------------------------------------------------------------------------------------

\begin{minipage}[t]{0.265\textwidth}
	\vspace{-\baselineskip} % Required for vertically aligning minipages
	
	\asect{Languages}
	
	\textbf{German} - native\\
	\textbf{English} - C2\\
\end{minipage}
\hfill
\begin{minipage}[t]{0.3\textwidth}
	\vspace{-\baselineskip} % Required for vertically aligning minipages
	
	\asect{Hobbies}
	
	cooking, baking, reading, writing, learning, sports
\end{minipage}
\hfill
\begin{minipage}[t]{0.3\textwidth}
	\vspace{-\baselineskip} % Required for vertically aligning minipages
	
	\asect{Interests}
	
	Linux, Rust, Go, neovim, fountain pens, physics, algorithms
\end{minipage}

%----------------------------------------------------------------------------------------

\end{document}
